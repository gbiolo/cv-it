\section{Esperienze professionali rilevanti}

\cvitem{Azienda}{Linear S.c.a.r.l. c/o TIM - Telecom Italia}
\cvitem{Settore}{Soluzioni e consulenze per il mercato IT}
\cvitem{Intervallo temporale}{07/2014 - in essere}
\cvitem{Ruolo}{Software Developer \& System Administrator}
\cvitem{Responsabilit\`a}{
    - Gestione appplicativa dei sistemi TIM riguardanti il billing \newline
    - Analisi di primo livello su eventuali problematiche applicative \newline
    - Preparazione ed installazione patch applicative \newline
    - Sviluppo tools di monitoraggio e reportistica
}
\cvitem{Descrizione}{
    il contratto \`e in continuazione del precedente, quindi le attivit\`a sui
    sistemi TIM e la creazione e manutenzione di tools di monitoraggio e reportistica
    sono le medesime. \newline
    Per l'azienda ho realizzato, su piattaforma Python, una web application per
    implementare un SMS Relay mediante modem USB. \newline
    Inoltre ho sviluppato un demone in Perl per i sistemi TIM (HP-UX, Linux e Solaris) per
    poter eseguire delle procedure automatizzate ed inviarne l'esito a Montrack, una
    piattaforma di monitoraggio e reportistica end-to-end di TIM realizzata da Reply.
}
\cvitem{Competenze}{HP-UX, Linux, BEA Weblogic, Oracle DB, Shell Scripting, Perl, Python}

\begin{center}\hfil\rule{0.75\textwidth}{.4pt}\hfil\end{center}

\cvitem{Azienda}{Aletea S.p.A. c/o TIM - Telecom Italia}
\cvitem{Settore}{Soluzioni e consulenze per il mercato IT}
\cvitem{Intervallo temporale}{09/2011 - 06/2014}
\cvitem{Ruolo}{Software Developer \& System Administrator}
\cvitem{Responsabilit\`a}{
    - Gestione appplicativa dei sistemi TIM riguardanti il billing \newline
    - Analisi di primo livello su eventuali problematiche applicative \newline
    - Preparazione ed installazione patch applicative \newline
    - Sviluppo tools di monitoraggio e reportistica
}
\cvitem{Descrizione}{
    le mansioni principali consistevano nella gestione applicativa dei sistemi, la
    preparazione ed installazione di patch applicative e la realizzazione di tools di
    monitoraggio applicativo ed automatizzazione di processi interni.\newline
    Per la realizzazione dei tools di monitoraggio ho utilizzato principalmente i
    linguaggi Perl e Python, ed alcuni casi scripting shell (principalmente KSH)
    ed AWK. \newline
    Ho inoltre collaborato allo sviluppo della piattaforma proprietaria di monitoraggio
    SIMM, lavorando principalmente nella componente di backend del sistema (realizzata
    in linguaggio Perl). \newline
    Per l’azienda ho inoltre realizzato, su piattaforma Perl, una web application per
    la gestione integrata di task aziendali e per la gestione delle risorse umane.
}
\cvitem{Competenze}{HP-UX, Linux, Tru64, BEA Weblogic, Oracle DB, Shell Scripting, Perl, Python}

\begin{center}\hfil\rule{0.75\textwidth}{.4pt}\hfil\end{center}

\cvitem{Azienda}{ArsLogica Sistemi s.r.l.}
\cvitem{Settore}{Soluzioni e consulenze per il mercato IT}
\cvitem{Intervallo temporale}{07/2010 - 10/2010}
\cvitem{Ruolo}{Stage universitario}
\cvitem{Responsabilit\`a}{
    - Studio teorico dei paradigmi di virtualizzazione e cloud computing \newline
    - Installazione e configurazione della piattaforma cloud computing OpenNebula
}
\cvitem{Descrizione}{
    lo stage universitario prevedeva lo studio a livello teorico dei paradigmi della
    virtualizzazione e del cloud computing, con l'analisi dei software disponibili per
    la virtualizzazione e le diverse tipologie di cloud computing. \newline
    Inoltre \`e stato affrontato il caso specifico di OpenNebula, con l'installazione
    e la configurazione di un ambiente cloud PaaS di test.
}
\cvitem{Competenze}{Debian Linux, Oracle VirtualBox, OpenNebula, Ruby}
