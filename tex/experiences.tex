\section{Esperienze professionali rilevanti}

\cvitem{Azienda}{Linear S.c.a.r.l. c/o TIM - Telecom Italia}
\cvitem{Settore}{Soluzioni e consulenze per il mercato IT}
\cvitem{Intervallo temporale}{07/2014 - in essere}
\cvitem{Ruolo}{Software Developer \& Application Analyst}
\cvitem{Responsabilit\`a}{
    - Gestione appplicativa dei sistemi TIM riguardanti il billing \newline
    - Analisi di primo livello su eventuali problematiche applicative \newline
    - Preparazione ed installazione patch applicative \newline
    - Sviluppo tools di monitoraggio e reportistica
}
\cvitem{Descrizione}{
    il contratto \`e in continuazione del precedente, quindi le attivit\`a sui sistemi TIM e di creazione e manutenzione di tools di monitoraggio e reportistica sono le medesime. \newline
    Per l'azienda ho realizzato, su piattaforma Python, una web application per implementare un SMS Relay mediante modem USB. \newline
    Inoltre ho sviluppato un demone in Perl da eseguire sui sistemi TIM per poter effettuare delle procedure automatizzate ed inviarne l'esito a Montrack, una piattaforma di monitoraggio e reportistica end-to-end di TIM realizzata da Reply.
}
\cvitem{Competenze}{HP-UX, Linux, BEA Weblogic, Oracle DB, Shell Scripting, Perl, Python}

\begin{center}\hfil\rule{0.75\textwidth}{.4pt}\hfil\end{center}

\cvitem{Azienda}{Aletea S.p.A. c/o TIM - Telecom Italia}
\cvitem{Settore}{Soluzioni e consulenze per il mercato IT}
\cvitem{Intervallo temporale}{09/2011 - 06/2014}
\cvitem{Ruolo}{Software Developer \& Application Analyst}
\cvitem{Responsabilit\`a}{
    - Gestione appplicativa dei sistemi TIM riguardanti il billing \newline
    - Analisi di primo livello su eventuali problematiche applicative \newline
    - Preparazione ed installazione patch applicative \newline
    - Sviluppo tools di monitoraggio e reportistica
}
\cvitem{Descrizione}{
    il ruolo ricoperto era di analista funzionale UNIX per i sistemi del prepagato TIM riguardanti il billing. Inoltre svolgevo il ruolo di sviluppatore Perl/Python per i prodotti di monitoraggio e reportistica interni. \newline
    Le mansioni principali consistevano nella gestione applicativa dei sistemi, la preparazione ed installazione di patch applicative, e la realizzazione di tools di monitoraggio applicativo ed automatizzazione di processi interni. \newline
    Nelle mie mansioni era prevista anche la gestione di VM su piattaforma Oracle WebLogic, con deploy di applicazioni in caso di installazioni di patch software o di nuove componenti. \newline
    Per la realizzazione dei tools di monitoraggio ho utilizzato principalmente i linguaggi Perl e Python, ed in minima parte scripting shell (principalmente KSH) ed AWK. \newline
    In diversi casi, per ottenere software di una certa complessit\`a (statistiche automatizzate, invio reportistica, tools di monitoraggio avanzato) ho realizzato delle librerie che permettessero di rendere il codice pulito, efficente e mantenibile. Tali librerie sono costituite da moduli Perl. \newline
    Ho inoltre collaborato allo sviluppo della piattaforma proprietaria di monitoraggio SIMM, lavorando principalmente nella componente di backend del sistema (realizzata interamente in linguaggio Perl). \newline
    Per l’azienda ho inoltre realizzato, su piattaforma Perl, una web application per la gestione integrata di task aziendali e per la gestione delle risorse umane.
}
\cvitem{Competenze}{HP-UX, Linux, Tru64, BEA Weblogic, Oracle DB, Shell Scripting, Perl, Python}

\begin{center}\hfil\rule{0.75\textwidth}{.4pt}\hfil\end{center}

\cvitem{Azienda}{ArsLogica Sistemi s.r.l.}
\cvitem{Settore}{Soluzioni e consulenze per il mercato IT}
\cvitem{Intervallo temporale}{07/2010 - 10/2010}
\cvitem{Ruolo}{Stage universitario}
\cvitem{Responsabilit\`a}{
    - studio teorico dei paradigmi di virtualizzazione e cloud computing \newline
    - installazione e configurazione della piattaforma cloud computing OpenNebula
}
\cvitem{Descrizione}{
    lo stage universitario prevedeva, in una prima fase, lo studio a livello teorico dei paradigmi della virtualizzazione e del cloud computing. Quindi ho analizzato le diverse soluzioni disponibili per la virtualizzazione, sia che si trattassero di tecnologie open source (XEN, KVM, ecc.) sia soluzioni a pagamento (IBM PowerVM, Parallels Virtuozzo, ecc.), e le diverse tipologie di cloud computing (SaaS, PaaS, ecc.). \newline
    In seguito \'e stato affrontato il caso specifico di OpenNebula, con l'installazione e la configurazione di un ambiente cloud di test (PaaS). \newline
}
\cvitem{Competenze}{Debian Linux, Oracle VirtualBox, OpenNebula, Ruby}
